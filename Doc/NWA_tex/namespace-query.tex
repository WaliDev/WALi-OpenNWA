\section{The \texttt{wali::nwa::query} namespace}
\label{Se:namespace-query}

The \texttt{wali::nwa::query} namespace provides functions related to
querying both the structure and language of an automaton.

In this section, we first discuss how to retrieve information from an
automaton's transitions (\sectref{query-transitions}) then move on to other
queries about an automaton proper.
Following that, we discuss queries about the
\textsl{language} of an automaton (\sectref{query-language}).


\subsection{Querying information about an automaton's transitions}
\label{Se:query-transitions}
The \texttt{wali::nwa::query} namespace provides a large number of functions
for retrieving information about an \texttt{NWA}'s transitions. The questions
answered by these functions are of the form, e.g., ``what are all symbols that
appear on a return transition where the call predecessor state is \texttt{s1}
and the return site state is \texttt{s1}?''

Functions return information either about all transitions or about
transitions of a particular type (internal, call, or return). The names of
these functions have regular patterns based on the information that is
known and desired, but the rules for producing the names are perhaps not as
simple as they could be. Because of this,
Tabs.~\ref{Ta:query-all-transitions}--\ref{Ta:query-return-transitions}
provide a quick reference for the functions.

The tables do not have \textsl{all} the information one needs to know in
order to call them, and the entries use some shorthands; but they provide enough
information for specifics to be
looked up in the header or Doxygen documentation. The caption of each table
gives the header that contains the functions listed. In the interest of
space, the types of the function arguments are omitted, but they are likely to
be what you expect. (The number of arguments is also always sufficient to
uniquely identify the function because all types are just \texttt{wali::Key}s
anyway.) Almost all functions return a \texttt{StateSet}, \texttt{SymbolSet},
or a \texttt{std::set} of pairs of a state and a symbol. (Those that return a
set of pairs are marked specially.)

%To use the table to answer the question above, we first look at the table for
%return transitions (\tableref{query-return-transitions}). We know the call
%predecessor and return site, so we find the corresponding row in the table
%(fourth from the bottom). We want to know all the symbols, so we look at the
%symbol column. We see there are two choices: \texttt{getReturnSym\_CallRet}
%and \texttt{getExits}. By looking at \texttt{returns.hpp} or Doxygen
%documentation, we can see that the former returns a \texttt{SymbolSet}, which
%is what we want. The latter returns
%a \texttt{std::set$<$std::pair$<$State,Symbol$>>$}, which contains the
%information we desire, but is likely to be less convenient to
%use. (The \texttt{State} component of each pair is the exit site, which is
%where the function name \texttt{getExits} comes from.)



%\newcolumntype{x}[1]{p{#1}}
\newcommand{\RP}{\tnote{1}} %"returns pair"

\setlength{\extrarowheight}{4pt}

\newgeometry{bottom=0.75in,top=0.75in}

\begin{sidewaystable}\sffamily
\begin{threeparttable}
\begin{tabular}{p{1in}p{1in}|@{\hspace{0.1in}}p{2.2in}p{2.2in}p{2in}}
\toprule\toprule
\multicolumn{2}{c|@{\hspace{0.1in}}}{\textsl{What you know}} & \multicolumn{3}{c}{\textsl{What you want}}                                                                 \tabularnewline
 this...        & ... and this      &    sources                    &   symbols                          &    targets                     \tabularnewline
\midrule
\midrule %-------------------------------------------------------------------------------------------------------------------------------
  source        &  (nothing)        &      ---                      &  (none)                            &  getSuccessors(nwa, src)       \tabularnewline
                &  symbol           &      ---                      &        ---                         &  getSuccessors(nwa, src, sym)  \tabularnewline
                &  target           &      ---                      &  getSymbol(nwa, src, tgt, \&sym)   &   ---                          \tabularnewline
\midrule %-------------------------------------------------------------------------------------------------------------------------------
  symbol        &  (nothing)        &                               &        ---                         &   (none)                       \tabularnewline
                &  target           & getPredecessors(nwa, sym, tgt)&       ---                         &   ---                          \tabularnewline
\midrule %-------------------------------------------------------------------------------------------------------------------------------
  target        &  (nothing)        & getPredecessors(nwa, tgt)     &  (none)                            &   ---                          \tabularnewline
\bottomrule\bottomrule
\end{tabular}
\caption{\textbf{Query functions for all transitions.} These functions are
  in the namespace \texttt{wali::nwa::query}; include the
  file \texttt{wali/nwa/query/transitions.hpp}. For return transitions, the
  ``source'' is the first component of the transition; nothing involving call
  predecessors (the second component of return transitions) appears in this table. A table
  entry of ``---'' means that the combination of arguments does not make sense. }
\end{threeparttable}
\label{Ta:query-all-transitions}
\end{sidewaystable}

\begin{sidewaystable}\sffamily
\begin{threeparttable}
\begin{tabular}{p{1in}p{1in}|@{\hspace{0.1in}}p{2.2in}p{2.2in}p{2in}}
\toprule\toprule
\multicolumn{2}{c|@{\hspace{0.1in}}}{\textsl{What you know}} & \multicolumn{3}{c}{\textsl{What you want}}                                                                 \tabularnewline
 this...        & ... and this      &    sources                    &   symbols                          &    targets                     \tabularnewline
\midrule
\midrule %-------------------------------------------------------------------------------------------------------------------------------
 (nothing)      &  (nothing)        & getSources(nwa)               &  getInternalSym(nwa)               &  getTargets(nwa)               \tabularnewline
\midrule %-------------------------------------------------------------------------------------------------------------------------------
  source        &  (nothing)        &      ---                      &  getInternalSym\_Source(nwa, src) \newline
                                                                       or getTargets(nwa, src)\RP        &  getTargets(nwa, src)\RP       \tabularnewline
                &  symbol           &      ---                      &        ---                         &  getTargets(nwa, src, sym)     \tabularnewline
                &  target           &      ---                      &  getInternalSym(nwa, src, tgt)     &   ---                          \tabularnewline
\midrule %-------------------------------------------------------------------------------------------------------------------------------
  symbol        &  (nothing)        & getSources\_Sym(nwa, sym)     &        ---                         &  getTargets\_Sym(nwa, sym)     \tabularnewline
                &  target           & getSources(nwa, sym, tgt)     &        ---                         &   ---                          \tabularnewline
\midrule %-------------------------------------------------------------------------------------------------------------------------------
  target        &  (nothing)        & getSources(nwa, tgt)\RP       &  getInternalSym\_Target(nwa, tgt) \newline
                                                                       or getSources(nwa, tgt)\RP        &   ---                          \tabularnewline
\bottomrule\bottomrule
\end{tabular}
\begin{tablenotes}
  \item[1] Returns a set of pairs (either source/symbol or symbol/target).
\end{tablenotes}
\caption{\textbf{Query functions for internal transitions.}
  These functions are in the namespace \texttt{wali::nwa::query};
  include the file \texttt{wali/nwa/query/internals.hpp}.
  A table entry of ``---'' means that combinations of arguments does not make
  sense.}
\end{threeparttable}
\label{Ta:query-internal-transitions}
\end{sidewaystable}

\begin{sidewaystable}\sffamily
\begin{threeparttable}
\begin{tabular}{p{1in}p{1in}|@{\hspace{0.1in}}p{2.2in}p{2.2in}p{2in}}
\toprule\toprule
\multicolumn{2}{c|@{\hspace{0.1in}}}{\textsl{What you know}} & \multicolumn{3}{c}{\textsl{What you want}}                                                                 \tabularnewline
 this...        & ... and this      &    call sites                   &   symbols                          &    entries                     \tabularnewline
\midrule
\midrule %-------------------------------------------------------------------------------------------------------------------------------
 (nothing)      &  (nothing)        & getCallSites(nwa)               &  getCallSym(nwa)                   &  getEntries(nwa)               \tabularnewline
\midrule %-------------------------------------------------------------------------------------------------------------------------------
 call site      &  (nothing)        &      ---                        &  getCallSym\_Call(nwa, call)\newline
                                                                         or getEntries(nwa, call)\RP       &  getEntries(nwa, call)\RP      \tabularnewline
                &  symbol           &      ---                        &        ---                         &  getEntries(nwa, call, sym)    \tabularnewline
                &  target           &      ---                        &  getCallSym(nwa, call, ent)        &   ---                          \tabularnewline
\midrule %-------------------------------------------------------------------------------------------------------------------------------
  symbol        &  (nothing)        & getCallSites\_Sym(nwa, sym)     &        ---                         &  getEntries\_Sym(nwa, sym)     \tabularnewline
                &  target           & getCallSites(nwa, sym, ent)     &        ---                         &   ---                          \tabularnewline
\midrule %-------------------------------------------------------------------------------------------------------------------------------
  entry         &  (nothing)        & getCallSites(nwa, ent)\RP       &  getCallSym\_Entry(nwa, ent) \newline
                                                                         or getCallSites(nwa, ent)\RP      &   ---                          \tabularnewline
\bottomrule\bottomrule
\end{tabular}
\begin{tablenotes}
  \item[1] Returns a set of pairs (either call site/symbol or symbol/entry).
\end{tablenotes}
\caption{\textbf{Query functions for call transitions.} These functions are in the
  namespace \texttt{wali::nwa::query}; include the
  file \texttt{wali/nwa/query/calls.hpp}. The ``call site'' is the source of the transition (and uses the argument 
  name \texttt{call}), and the ``entry'' of the transition is the target (and
  uses the argument name \texttt{ent}). }
\end{threeparttable}
\label{Ta:query-call-transitions}
\end{sidewaystable}

\begin{sidewaystable}\footnotesize\sffamily
\begin{threeparttable}
\begin{tabular}{p{0.6in}p{0.65in}p{0.6in}|@{\hspace{0.1in}}p{1.75in}p{1.9in}p{1.9in}p{2in}}
\toprule\toprule
\multicolumn{3}{c|@{\hspace{0.1in}}}{\textsl{What you know}}  & \multicolumn{4}{c}{\textsl{What you want}}                                                                                                                                                \tabularnewline
 this...  & and this ...  & and this & exit sites & call predecessors &
 symbols & return sites \tabularnewline
\midrule
\midrule %--------------------------------------------------------------------------------------------------------------------------------------------------------------------------------------------------------------------
 (nothing)      &  (nothing)        &  (nothing)    & getExits(nwa)                 &  getCalls(nwa)                        &  getReturnSym(nwa)                           &  getReturns(nwa)                            \tabularnewline
\midrule %--------------------------------------------------------------------------------------------------------------------------------------------------------------------------------------------------------------------
 exit site      &  (nothing)        &  (nothing)    &      ---                      &  getCalls\_Exit(nwa, exit)\RP         &  getReturnSym\_Exit(nwa, exit) \newline
                                                                                                                               or getReturns\_Exit(nwa, exit)\RP\newline
                                                                                                                               or getCalls\_Exit(nwa, exit)\RP             &  getReturns\_Exit(nwa, exit)\RP             \tabularnewline
                \cline{2-7} %-------------------------------------------------------------------------------------------------------------------------------------------------------------------------------------------------
                &  call pred        &  (nothing)    &      ---                      &    ---                                &  getReturnSym\_ExitCall(nwa, exit, \newline
                                                                                                                               \phantom{getReturnSym\_ExitCall(}call) \newline
                                                                                                                               or getReturns(nwa, exit, call)\RP           &  getReturns(nwa, exit, call)\RP             \tabularnewline
                &                   &  symbol       &      ---                      &    ---                                &        ---                                   &  getReturns(nwa, exit, call, sym)           \tabularnewline
                &                   &  return       &      ---                      &    ---                                &  getReturnSym(nwa, exit, call, \newline
                                                                                                                               \phantom{getReturnSym(}ret)                 &    ---                                      \tabularnewline
                \cline{2-7} %-------------------------------------------------------------------------------------------------------------------------------------------------------------------------------------------------
                &  symbol           &  (nothing)    &      ---                      &  getCalls\_Exit(nwa, exit, sym)       &        ---                                   &  getReturns\_Exit(nwa, exit, sym)           \tabularnewline
                &                   &  return       &      ---                      &  getCalls(nwa, exit, sym, ret)        &        ---                                   &  getEntries(nwa, call, sym, ret)            \tabularnewline
                \cline{2-7} %-------------------------------------------------------------------------------------------------------------------------------------------------------------------------------------------------
                &  return site      &  (nothing)    &      ---                      &  getCalls(nwa, exit, ret)\RP          &  getReturnSym\_ExitRet(nwa, exit, \newline
                                                                                                                               \phantom{getReturnSym\_ExitRet(}ret) \newline
                                                                                                                               or getCalls(nwa, exit, ret)\RP              &   ---                                       \tabularnewline
\midrule %-------------------------------------------------------------------------------------------------------------------------------------------------------------------------------------------------------------------
 call pred      &  (nothing)        &  (nothing)    & getExits\_Call(nwa, call)\RP  &   ---                                 &  getReturnSym\_Call(nwa, call) \newline
                                                                                                                               or getReturns\_Call(nwa, call)\RP\newline
                                                                                                                               or getExits\_Call(nwa, call)\RP            &  getReturnSites(nwa, call) or \newline
                                                                                                                                                                             getCallSuccessors(nwa, call) or \newline
                                                                                                                                                                             getReturns\_Call(nwa, call)\RP              \tabularnewline
                \cline{2-7} %-------------------------------------------------------------------------------------------------------------------------------------------------------------------------------------------------
                &  symbol           &  (nothing)    & getExits\_Call(nwa, call, sym)&   ---                                 &        ---                                   &  getCallSuccessors(nwa, call, sym) \newline
                                                                                                                                                                              or getReturns\_Call(nwa, call, sym)        \tabularnewline
                &                   &  return       & getExits(nwa, call, sym, ret) &   ---                                 &        ---                                   &    ---                                      \tabularnewline
                \cline{2-7} %-------------------------------------------------------------------------------------------------------------------------------------------------------------------------------------------------
                &  return site      &  (nothing)    & getExits(nwa, call, ret)\RP   &   ---                                 &  getReturnSym\_CallRet(nwa, call, \newline
                                                                                                                               \phantom{getReturnSym\_CallRet(}ret) \newline
                                                                                                                               or getExits(nwa, call, ret)\RP              &   ---                                       \tabularnewline
\midrule %--------------------------------------------------------------------------------------------------------------------------------------------------------------------------------------------------------------------
 symbol         &  (nothing)        &  (nothing)    & getExits\_Sym(nwa, c)         &  getCalls\_Sym(nwa, c)                &   ---                                        &  getReturns\_Sym(nwa, sym)                  \tabularnewline
                \cline{2-7} %-------------------------------------------------------------------------------------------------------------------------------------------------------------------------------------------------
                &  return site      &  (nothing)    & getExits\_Ret(nwa, call, ret) &  getCallPredecessors(nwa, sym, ret) \newline
                                                                                       or getCalls\_Ret(nwa, sym, c)        &   ---                                        &   ---                                       \tabularnewline
\midrule %--------------------------------------------------------------------------------------------------------------------------------------------------------------------------------------------------------------------
 return site    &  (nothing)         & (nothing)    & getExits\_Ret(nwa, ret)       &  getCallPredecessors(nwa, ret) \newline
                                                                                       or getCalls\_Ret(nwa, ret)\RP        &  getReturnSym\_Ret(nwa, ret) or \newline
                                                                                                                               getCalls\_Ret(nwa, ret)\RP                  &   ---                                       \tabularnewline
\bottomrule\bottomrule
\end{tabular}
\begin{tablenotes}
  \item[1] Returns a set of pairs (a symbol with one of the states, in the order of the raw transition).
\end{tablenotes}
\caption{\textbf{Query functions for return transitions.} These functions are in the
  namespace \texttt{wali::nwa::query}; include the
  file \texttt{wali/nwa/query/returns.hpp}.   The ``exit site'' is the source of the transition
  (the first component) and uses the argument name \texttt{exit} in this table;
  the ``call predecessor'' is the second component and uses the argument
  name \texttt{call}; the symbol is the third component and uses the argument
  name \texttt{sym}; the ``return site'' is the fourth component and uses the
  argument name \texttt{ret}. }
\end{threeparttable}
\label{Ta:query-return-transitions}
\end{sidewaystable}
\restoregeometry


\subsection{Querying other physical aspects of an automaton}
\label{Se:query-automaton}

The NWA library has two additional functions for querying attributes
of NWAs themselves. One computes whether an NWA is deterministic, and
the other computes whether two automata have any states in common.

These functions are declared in the header
\texttt{wali/nwa/query/automata.hpp} (and, of course, defined in the
namespace \texttt{wali::nwa::query}).

\begin{description}
  \item[\texttt{bool isDeterministic(NWA const \& nwa)}]
    Computes whether the given NWA is deterministic and returns the
    result. \textit{Warning:} The result is not cached in the NWA, and
    as presently-implemented, this function is very
    inefficient. Improvements to this will come in a later version of
    the library.
  \item[\texttt{bool statesOverlap(NWA const \& a, NWA const \& b)}]
    Comptuers whether the two NWAs share any states and returns the
    result.
\end{description}


\subsection{Querying properties of an automaton's language}
\label{Se:query-language}

The NWA library has several functions for querying standard
formal-language theoretic properties, and in addition supports
the poststar and prestar queries used in program analysis.

The functions that perform standard language-theoretic queries are
located in namespace \texttt{wali::nwa::query} and are declared in
\texttt{wali/nwa/query/language.hpp}. They are:
\begin{description}
  \item[\texttt{bool languageContains(NWA const \& nwa, Nested Word
    const \& word)}] Computes whether \texttt{word} is a member of
    \texttt{nwa}'s language, and returns the result. (It simulates the
    NWA in a nondeterministic fashion; the NWA is not determinized to answer
    this query.)
  \item[\texttt{bool languageIsEmpty(NWA const \& nwa)}] Computes
    whether the NWA's language is empty, and returns the result.
  \item[\texttt{bool languageSubsetEq(NWA const \& left, NWA const \& right)}]
    Computes whether $L(\texttt{left}) \subseteq L(\texttt{right})$
    and returns the result.
  \item[\texttt{bool languageEquals(NWA const \& left, NWA const \& right)}]
    Equivalent to \texttt{languageSubsetEq(left, right) \&\&
    languageSubsetEq(right, left)}
\end{description}

Finally, there are functions for performing a poststar and prestar
query on an NWA. These work by converting the NWA to a WPDS (see
\sectref{Conversions} and, in particular, \sectref{NWAtoPDS}) using
the function \texttt{NWAtoPDScalls} then running the query on the PDS.

There are two variants of each. One returns the WFA that results from
the query, and the other stores the WFA in an output parameter. The
\texttt{WFA} type is \texttt{wali::wfa::WFA}.

The functions that perform poststar and prestar are in the namespace
\texttt{wali::nwa::query} and are declared in the header
\texttt{wali/nwa/query/weighted.hpp}. They are:
\begin{description}
  \item[\texttt{WFA prestar( NWA const \& nwa, WFA \& input, WeightGen \& wg )}]
  \item[\texttt{void prestar( NWA const \& nwa, WFA \& input, WFA \& output, WeightGen \& wg )}]
    Computes the prestar of the configurations specified by
    \texttt{input}, using the weights generated by \texttt{wg}. Either
    returns the result or stores it in the parameter \texttt{output}.
  \item[\texttt{WFA poststar( NWA const \& nwa, WFA \& input, WeightGen \& wg )}]
  \item[\texttt{void poststar( NWA const \& nwa, WFA \& input, WFA \& output, WeightGen \& wg )}]
    Computes the poststar of the configurations specified by
    \texttt{input}, using the weights generated by \texttt{wg}. Either
    returns the result or stores it in the parameter \texttt{output}.
\end{description}

\TODO{Check to see if I can make input and wg const.}
